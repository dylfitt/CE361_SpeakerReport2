\documentclass[12pt]{article}
\usepackage[letterpaper,margin=1in]{geometry}
%\usepackage{multicol}
%\usepackage{tabularx}
%\usepackage{array}
%\usepackage{comment}

\pagestyle{empty}

\begin{document}
	\begin{flushright}
		Dylan Fitt\\
		CE 361\\
		3/15/2023
	\end{flushright}
	\begin{centering}
		\subsection*{Speaker Report 2: Kordel Braley}
	\end{centering}
	
%	Kordel Braley talked about a small variety of things relating 
%	to the transportation industry.
%	Some things he talked about include
%	using VSIM to simulate transportation impacts,
%	how transportation issues are often a result of land use decisions
%	that were made by politicians in favor of cars,
%	how we need to try to fit the constraints of the project we are
%	doing,
%	and that we need to understand why people do what they do.
%	He said several times that part of the reason that we build public
%	transportation is to give people options.

	While speaking to us in class,
	there were a few things that Kordel Braley said that caught my
	attention.
	One idea that he repeated several times was that part of the reason
	that we build public transportation is to give people options.
	I think that people are unlikely to leave their cars
	unless they are given an incentive to do otherwise.
	
	Sometimes, that incentive isn't planned.
	My in-laws are used to having two cars,
	but right now they don't have any usable ones.
	They have an SUV that they drive for most purposes
	that can fit the whole family
	and had an old, run-down sedan that the kids drove to work and
	school.
	Just a few months ago,
	their sedan was involved in a crash and it was totaled.
	The family got around fine with one car
	because when my father-in-law commutes to Salt Lake,
	my mother-in-law drops him off at the FrontRunner station
	and he takes the train to work.
	However,
	when the steering on the SUV gave out last week,
	my wife's family suddenly found themselves stranded in their suburban
	island of a neighborhood.
	The nearest transit stop isn't all that far away,
	but it has an impractical route
	that my father-in-law chose to work from home
	and my mother-in-law will be walking to her work.
	They are lucky that they even had these options to work around
	this problem.
	
	Others in similar situations may not be fortunate enough to be able
	to work from home or walk to work and school.
	This is exactly why we need to give options to people,
	including public transit, bicycle infrastructure, and walkable areas.
	This can become a problem when many of our roads
	and most of our transportation system is designed primarily for cars.
	However, by understanding why people are doing what they do,
	as Kordel said, we can improve service for more people.
	
	The project going down 200 South in Salt Lake City is a good example
	of this.
	By improving the quality of service for people riding buses down
	this corridor
	we are giving people more options,
	moving a baby-step away from car-centric roads,
	and showing that we understand those that choose to or are forced
	to ride UTA's buses.
	I hope that we can continue to give people options
	and that we can continue to learn about the people that use our
	transportation systems.
	
\end{document}